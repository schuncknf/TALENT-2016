\documentclass[a4paper,10pt]{article}
\usepackage[utf8]{inputenc}

%opening
\title{WEEK2: 3D Wood-Saxon potential for $^{208}$Pb.}
\author{Ni Fang\\ Kai Wang\\Claudia Gonzalez-Boquera}
\date{TALENT 2016}
\begin{document}

\maketitle
\textbf{CONSIDER FOLDER NAMED COULOMB, THE FOLDER WITHOUT COULOMB IS OUTDATED.}
\section{Structure}

This program calculates the eigenvalues and wavefunctions of each given nlj orbital, 
with Wood-Saxon and spin-orbit potentials. In addition we have put the Coulomb potential 
when studying protons.
The program consists in a loop over isospin (ipart=1,2 first neutrons, then protons), 
and calculates for each possible nlj (up to the limit put in the input file) the 
corresponding eigenvalue and wavefunction.

The eigenvalues are found with the bissection method, which includes the numerov 
method in order to calculate the eigenfunction for each eigenvalue. For each
eigenfunction found, we calculate its nodes, and we modify the extremes of the 
bissection method according to them. When the extremes are closer than epsi,
we exit the loop, and put the eigenvalue as the right extreme of the bissection
method. 
Then we recalculate the eigenfunction with the shooting method, with 
right hand side equal to 0, in order to find the convergence, and we normalize it. 

When all possible orbitals are calculated, we rearrange the energy in ascending order, 
in order to find the correct order in the levels. 

After, the density profiles are calculated.
\subsection{Functions and Subroutines}
\begin{itemize}
 \item Function Vpot(xy, mm, ip) :  xy: point in the mesh to study, mm: spin, ip: neutrons/protons.
 Function that calculates the total potential. 
 Includes the Wood-Saxon potential, the Spin-Orbit contributions and the Coulomb potential
 when studying the protons. For the equations in order to calculate each potential, refer to manuale$\_$hf.pdf. 

 \item Subroutine simpson(N, dx, f, res) : subroutine to integrate the densities, in order to normalize the 
eigenfunctions. Inputs: N number of points in the mesh, dx space between the points in the mesh,f array
to integrate. Output:res  value of the integration. 
\item Subroutine arrange$\_$energy (matrix, numorb, wavef): Subroutine that arranges the energy in 
ascending order. As the energy is given as an input inside matrix (includes energy and quantum numbers), 
all is arranged according to the value of the energy. Also the matrix that contains all the wavefunctions 
(wavef) is arranged according to the same pattern in the energies. Numorb is the number of orbitals calculated.

\end{itemize}

\section{Files}
\begin{itemize}
 \item global.f90: module that includes the global variables. 
 \item WS.input: input file (see below).
 \item main.f90: main file that includes the code to find the eigenvalues and eigenfunctions. 
 \item Makefile : makefile to compile the code (see below). 
 \item Simpson.f90: file that contains the subroutine that integrates using the Simpson method.
 \item arrange.f90: file that contains the subroutine that arranges the energy in ascending order. 
\end{itemize}


\section{Input file}
\begin{itemize}
 \item NN and ZZ = number of neutrons and protons of the nucleus to study. 
 \item epsi= epsilon error to find the solution. 
 \item h2m = value of $\hbar^2/2m$.
 \item e= value of e$^2$, value of the charge for the Coulomb potential.
 \item R$_\mathrm{box}$ = size of our considered and studied system.
 \item h= space between two points in the mesh.
 \item E$_\mathrm{minus}$= left value of the energy when finding the eigenvalues using the bissection method.
 \item E$_\mathrm{plus}$ = right value of the energy when finding the eigenvalues using the bissection method.
 \item n$\_$max, l$\_$max = maximum n and l quantum numbers to calculate.
 \item r0 = parameter to calculate R0 in Wood-Saxon potential. 
 \item a = difusivity of the Wood-Saxon potential. 
 \item out$\_$wave$\_$func = logical parameter to output all the wavefunctions calculated in files or not.
\end{itemize}

\section{Output}
\begin{itemize}
 \item Energy levels of neutrons (energy$\_$level$\_$n.dat) and protons (energy$\_$level$\_$p.dat).
 \item Density profiles of neutrons (density$\_$N.dat), protons(density$\_$P.dat) and total (density$\_$T.dat)
 \item If wanted, output with all wavefunctions calculated. (out$\_$wave$\_$func =.true.)
\end{itemize}



\section{Makefile}
To run the program, there is a Makefile, that compiles all files that form the code. 
To compile, type \textbf{make}, and an executable \textbf{main} is created. To run the code, 
type \textbf{./main}. And to delete all exectuables and outputs, type either \bf{make clean} or
\textbf{make clear}. Makefile has to be changed with vi or vim programs.

\end{document}
